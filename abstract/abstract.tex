\begin{abstract}

本文给出了大连大学硕士学位论文的写作规范和排版格式要求。文中格式可作为编排硕士学位论文的格式模板,供研究生参考使用。

摘要部分说明:

“摘要”是摘要部分的标题,不可省略。

标题“摘要”选用模板中的样式所定义的“标题1”,再居中;或者手动设置成字体:黑体,居中,字号:小三,1.5倍行距,段后11磅,段前为0。

论文摘要是学位论文的缩影,文字要简练、明确。内容要包括目的、方法、结果和结论。单位制一律换算成国际标准计量单位制,除特别情况外,数字一律用阿拉伯数码。文中不允许出现插图。重要的表格可以写入。

摘要正文选用模板中的样式所定义的“正文”,每段落首行缩进2个汉字;或者手动设置成每段落首行缩进2个汉字,字体:宋体,字号:小四,行距:多倍行距 1.25,间距:前段、后段均为0行,取消网格对齐选项。

篇幅以一页为限,字数为600-800字。

摘要正文后,列出3-5个关键词。“关键词:”是关键词部分的引导,不可省略。关键词请尽量用《汉语主题词表》等词表提供的规范词。

关键词与摘要之间空一行。关键词词间用分号间隔,末尾不加标点,3-5个,黑体,小四,加粗。


\keywords{写作规范;排版格式;硕士学位论文}

\end{abstract}


