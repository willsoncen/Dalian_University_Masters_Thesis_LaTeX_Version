\bichapter{\LaTeX{}的基本概念}{\LaTeX{} Basics}
\label{chap:intro}

\bisection{概述}{Introduction}

\bisubsection{\TeX{}}{\TeX{}}

\TeX{} 是高德纳 (Donald E.~Knuth) 为排版文字和数学公式而开发的软件。
1977 年,正在编写《计算机程序设计艺术》的高德纳意识到每况愈下的排版质量将影响其著作的发行,
为扭转这种状况,他着手开发 \TeX{},发掘当时刚刚用于出版工业的数字印刷设备的潜力。
1982 年,高德纳发布 \TeX{} 排版引擎,而后在 1989 年又为更好地支持 8-bit 字符和多语言排版而予以改进。
\TeX{} 以其卓越的稳定性、跨平台能力和几乎没有 bug 的特性而著称。
它的版本号不断趋近于 $\pi$,当前为 3.141592653。

\TeX{} 读作“Tech”,与汉字“泰赫”的发音相近,其中“ch” 的发音类似于“h”。\TeX{} 的拼写来自希腊词语
τεχνική (technique,技术) 开头的几个字母,在 ASCII 字符环境中写作 \texttt{TeX}。

\bisubsection{\LaTeX{}}{\LaTeX{}}

\index{LaTeX@\LaTeX}
\index{LaTeX2e@\LaTeXe}
\LaTeX{} 是一种使用 \TeX{} 程序作为排版引擎的格式(format),可以粗略地将它理解成是对 \TeX{} 的一层封装。
\LaTeX{} 最初的设计目标是分离内容与格式,以便作者能够专注于内容创作而非版式设计,并能以此得到高质量排版的作品。
\LaTeX{} 起初由 Leslie Lamport 博士开发,目前由 \LaTeX{} 工作组%
\footnote{\url{https://www.latex-project.org}}进行维护。

\LaTeX{} 读作“Lah-tech” 或者“Lay-tech”,与汉字“拉泰赫”或“雷泰赫”的发音相近,在 ASCII 字符环境写作 \texttt{LaTeX}。
\LaTeXe{} 是 \LaTeX{} 的当前版本,意思是超出了第二版,但还远未达到第三版,在 ASCII 字符环境写作 \texttt{LaTeX2e}。

\bisubsection{\LaTeX{} 的优缺点}{Advantages and disadvantages}

经常有人喜欢对比 \LaTeX{} 和以 Microsoft Office Word 为代表的“所见即所得”%
(What You See Is What You Get)字处理工具。
这种对比是没有意义的,因为 \TeX{} 是一个排版引擎,\LaTeX{} 是其封装,而 Word 是字处理工具。
二者的设计目标不一致,也各自有自己的适用范围。

不过,这里仍旧总结 \LaTeX{} 的一些优点:

\begin{enumerate}
\item 具有专业的排版输出能力,产生的文档看上去就像“印刷品”一样。
\item 具有方便而强大的数学公式排版能力,无出其右者。
\item 绝大多数时候,用户只需专注于一些组织文档结构的基础命令,无需(或很少)操心文档的版面设计。
\item 很容易生成复杂的专业排版元素,如脚注、交叉引用、参考文献、目录等。
\item 强大的可扩展性。世界各地的人开发了数以千计的 \LaTeX{} 宏包用于补充和扩展 \LaTeX{} 的功能。
\item 能够促使用户写出结构良好的文档——而这也是 \LaTeX{} 存在的初衷。
\item \LaTeX{} 和 \TeX{} 及相关软件是跨平台、免费、开源的。
无论用户使用的是 Windows,macOS(OS X),GNU/Linux 还是 FreeBSD 等操作系统,都能轻松获得和使用这一强大的排版工具,并且获得稳定的输出。
\end{enumerate}

\LaTeX{} 的缺点也是显而易见的:

\begin{enumerate}
\item 入门门槛高。
\item 不容易排查错误。\LaTeX{} 作为一个依靠编写代码工作的排版工具,其使用的宏语言比 C++ 或 Python 等程序设计语言在错误排查方面困难得多。
它虽然能够提示错误,但不提供调试的机制,有时错误提示还很难理解。
\item 不容易定制样式。\LaTeX{} 提供了一个基本上良好的样式,为了让用户不去关注样式而专注于文档结构。
但如果想要改进 \LaTeX{} 生成的文档样式则是十分困难的。
\item 相比“所见即所得”的模式有一些不便,为了查看生成文档的效果,用户总要不停地编译。
\end{enumerate}
